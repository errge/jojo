\documentclass[a4paper,12pt]{article}
\usepackage[utf8]{inputenc}
\usepackage{amsmath, amssymb}
\usepackage{enumitem}
\usepackage{multicol}
\usepackage[a4paper, margin=25mm]{geometry} % smaller margins
\pagestyle{empty}

\begin{document}

Matekórára hozni:
\begin{itemize}
\item Tablet és laptop is, hogy lehessen jól rajzolni
\item Matekfüzet az óráról, kérdések, mit tanultatok?
\item Feladatok ami nem megy vagy nem értetted órán
\item Feladatgyűjtemény
\item Iskolai házifeladat megoldásmenete
\item Gergő+Éva házifeladat megoldásmenete
\end{itemize}

\renewcommand{\arraystretch}{1.5}

\begin{enumerate}
%  \setcounter{enumi}{10}
\item Számoljuk ki az alábbi számokat! Mindegyik feladatnál lehet ötletelni, ne használj számológépet! Csak az eredményt ellenőrizzük számológéppel!

  \begin{tabular}{@{}p{0.2cm}p{0.5\textwidth}p{0.2cm}p{0.5\textwidth}@{}}

    a) & ${17\cdot59 \over 59\cdot42} = ?$ & b) & lnko$(17\cdot59,59\cdot42) = ?$ \\

    c) & ${17+59 \over 59+42} = ?$         & d) & lnko$(17+59,59+42) = ?$ \\

    e) & ${17\cdot59 \over 59+42} = ?$     & f) & lnko$(17\cdot59,59+42) = ?$ \\

    g) & ${17+59 \over 59\cdot42} = ?$     & h) & lnko$(17+59,59\cdot42) = ?$ \\

  \end{tabular}

\item Mik az ismeretlenek lehetséges értékei az alábbi feladatokban?

  \begin{tabular}{@{}p{0.2cm}p{0.2\textwidth}p{0.2cm}p{0.3\textwidth}@{}p{0.2cm}p{0.2\textwidth}@{}}
  a) & $17(x - 5) = 51$ \\
  b) & ${3x \over 3\cdot 7} = 2$ & c) & ${3+x \over 3+7} = 2$ \\
  d) & ${(x+5)^2 \over 2} = 8$ & e) & ${x^2+10x+25\over 2} = 2$ & f) & ${x^2-10x+25\over 2} = 2$ \\
  g) & ${(x-5)^2-8} = 41$ & h) & $x^2-10x+17 = 41$ \\
  i) & ${{x^4 - 256}\over {x^2 - 16}} = 8x + 25$
  \end{tabular}

\end{enumerate}

\end{document}
